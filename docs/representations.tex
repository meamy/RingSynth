% --------------------------------------------------------------------
\documentclass{article}

% ----------------------------------------------------------------------
% Packages.
\usepackage{amsmath,amsfonts,amsthm,amssymb}
\usepackage{braket}

% ----------------------------------------------------------------------
% Some notation.
\newcommand{\Z}{\mathbb{Z}}
\newcommand{\N}{\mathbb{N}}
\newcommand{\R}{\mathbb{R}}
\newcommand{\C}{\mathbb{C}}
\newcommand{\D}{\mathbb{D}}
\newcommand{\norm}[1]{\left\lVert#1\right\rVert}
\newcommand{\inv}{^{-1}}
\newcommand{\dagg}{^\dagger}
\newcommand{\s}[1]{\{#1\}}
\newcommand\restrict[1]{\raisebox{-.5ex}{$|$}_{#1}}
\newcommand{\diag}{\operatorname{diag}}
\newcommand{\ove}{\overline{e}}

% ----------------------------------------------------------------------
% Theorem like environments.
\theoremstyle{plain}
\newtheorem{theorem}{Theorem}[section]
\newtheorem{lemma}[theorem]{Lemma}
\newtheorem{proposition}[theorem]{Proposition}
\newtheorem{corollary}[theorem]{Corollary}
\theoremstyle{definition}
\newtheorem{definition}[theorem]{Definition}
\newtheorem{example}[theorem]{Example}
\theoremstyle{remark}
\newtheorem{remark}[theorem]{Remark}

% ----------------------------------------------------------------------
% Macros for bibliography.
\newcommand{\urlalt}[2]{\href{#2}{\nolinkurl{#1}}}
\newcommand{\arxiv}[1]{\urlalt{arXiv:#1}{http://arxiv.org/abs/#1}}


% --------------------------------------------------------------------
\title{} \author{} \date{}

% --------------------------------------------------------------------
\begin{document}

% --------------------------------------------------------------------
\section{Ring Extensions}

Throughout, rings have a multiplicative identity and ring
homomorphisms preserve it.

\begin{definition}
  If $R$ and $S$ are rings such that $R\subseteq S$ then $S$ is an
  \emph{extension} of $R$.
\end{definition}

\begin{definition}
  Let $R \subseteq S$ be an extension of rings and let $M$ be a subset
  of $S$. The \emph{extension of $R$ generated by $M$}, denoted
  $R[M]$, consists of all the elements of $S$ that can be written as
  \[
  \sum_{j_i \in \N} a_{j_1,\ldots,j_n}c_1^{j_1}\cdots c_n^{j_n}
  \]
  where $a_{j_1, \ldots, j_n}\in R$, $c_i \in M$, and only finitely
  many of the $a_{j_1,\ldots, j_n}$ are not 0. If $M=\s{c_1,\ldots
    c_n}$ is a finite subset of $S$ then $R[M]$ is denoted
  $R[c_1,\ldots, c_n]$.
\end{definition}

\begin{definition}
  Let $R \subseteq S$ be an extension of rings. Then $S$ is a
  \emph{finite extension} of $R$ if $S=R[c_1,\ldots c_n]$ for some
  $c_1, \ldots, c_n\in S$.
\end{definition}

\begin{definition}
  Let $R\subseteq S$ be an extension of rings. An element $s$ of $S$
  is \emph{integral over $R$} if there exists a monic polynomial $p$
  with coefficients in $R$ such that $p(s)=0$. If every $s\in S$ is
  integral over $R$ then we say that $S$ is an \emph{integral
    extension} of $R$.
\end{definition}

\begin{proposition}
  Let $R\subseteq S$ be an extension of rings. Then $S$ is a finite
  extension of $R$ if and only if $S=R[c_1,\ldots, c_n]$ for elements
  $c_1, \ldots, c_n$ of $S$ that are integral over $R$.
\end{proposition}  

\begin{proposition}
  Let $R\subseteq S \subseteq T$ be extensions of rings.
  \begin{itemize}
  \item If $R\subseteq S$ and $S\subseteq T$ are finite then
    $R\subseteq T$ is finite.
  \item If $R\subseteq S$ and $S\subseteq T$ are integral then
    $R\subseteq T$ is integral.
  \end{itemize}
\end{proposition}

\begin{example}
  Let $R$ be a subring of $\C$. Then $\Z\subseteq R$ so if $c\in\C$ is
  the root of a monic polynomial with integer coefficients then $c$ is
  integral over $R$. In particular, roots of unity and square roots of
  integers are integral over any subring of $\C$.
\end{example}

% --------------------------------------------------------------------
\section{Embeddings}

We write $R^{n\times n}$ for the ring of $n\times n$ matrices and $I$
for its multiplicative identity, using $I_n$ in the cases where we
want to make the dimension explicit. We denote the tensor product of
two matrices $M$ and $N$ by $M\otimes N$. If $r$ is an element of a
ring $R$ we will sometimes write $r$ for the matrix $rI \in R^{n\times
  n}$. Note that $r=rI=r\otimes I$.

We use block matrices and freely move between the block form and the
flattened form of a matrix when it is convenient. In particular, we do
not distinguish between the rings $R^{nk\times nk}$, $(R^{n\times
  n})^{k\times k}$, and $(R^{k\times k})^{n\times n}$.

\begin{definition}
  \label{def:kemb}
  Let $R\subseteq S$ be an extension of rings. A \emph{$k$-embedding}
  of $S$ into $R$ is a ring homomorphism $e:S\to R^{k\times k}$ such
  that $e(r) = r\otimes I_k$ for every $r\in R$.
\end{definition}

% Asking for $e(r) = r\otimes I_k$ for every $r\in R$ makes lifting
% arguably more natural but requiring $e(r) = rI_k$ would also
% possible.

We think of a $k$-embedding $e:S\to R^{k\times k}$ as an encoding of
the elements of $S$ as $k$-dimensional matrices over $R$. To encode
matrices over $S$, rather than elements of $S$, we lift $e$ from $S$
to $S^{n\times n}$ by applying the embedding component-wise.

\begin{definition}
  \label{def:klift}
  Let $e:S\to R^{k\times k}$ be a $k$-embedding. The \emph{lifting of
    $e$ to $S^{n\times n}$} is the function $\ove :S^{n\times n} \to
  R^{nk\times nk}$ defined by $(\ove (M))_{i,j} = e(M_{i,j})$ for
  every $M\in S^{n\times n}$ and every $1\leq i,j\leq n$.
\end{definition}

In Definition~\ref{def:klift}, the matrix $\ove (M)$ is defined as an
$n\times n$ block matrix whose blocks are $k\times k$ matrices with
entries in $R$. In keeping with the convention discussed above, we
interpret $\ove (M)$ as an $nk \times nk$ matrix with entries in
$R$. A tedious but straightforward calculation shows that $\ove $ is
itself a $k$-embedding.

\begin{proposition}
  \label{prop:lift}
  Let $R\subseteq S$ be an extension of rings. If $e:S\to R^{k\times
    k}$ is a $k$-embedding then its lifting $\ove :S^{n\times n} \to
  R^{nk\times nk}$ is a $k$-embedding as well.
\end{proposition}

\begin{corollary}
  \label{cor:composition}
  If $R\subseteq S\subseteq T$ be extensions of rings such that $T$
  $k$-embeds into $S$ and $S$ $\ell$-embeds into $R$, then $T$
  $k\ell$-embeds into $R$.
\end{corollary}

\begin{proof}
  Let $e:T \to S^{k \times k}$ and $f:S\to R^{\ell \times \ell}$ be
  the two embeddings. By Proposition~\ref{prop:lift} we can lift $f$
  to a $k$-embedding $\overline{f}:S^{k\times k} \to R^{k\ell \times
    k\ell}$. Then $\overline{f}\circ e : T \to R^{k\ell \times k\ell}$
  is a ring homomorphism and
  \[
  \overline{f}\circ e (r) = \overline{f}(r\otimes I_k) = r\otimes I_k
  \otimes I_\ell = r\otimes I_{k\ell}
  \]
  as required.
\end{proof}

% --------------------------------------------------------------------
\section{Embedding Integral Elements}

In this section, $R\subseteq S$ is an extension of rings and $s\in S$
is integral over $R$. We write $p$ for the minimal polynomial of
$s$. (Is it unique? If not, does it matter?) Moreover, we assume that
there exists $d\in \N$ such that $\s{s^0, \ldots s^d}$ forms a basis
for $R[s]$ as an $R$-module. Presumably, $d\leq \deg(p)$.

\begin{definition}
  A \emph{$k$-representation} of $s$ is a matrix $M\in R^{k\times k}$
  such that $p(M)=0$.
\end{definition}

\begin{proposition}
  Any $k$-representation of $s$ extends to a $k$-embedding $R[s]\to
  R^{k\times k}$.
\end{proposition}

\begin{proof}
  Let $M$ be a $k$-representation of $s$. Since $\s{s^0, \ldots, s^d}$
  forms a basis for $R[s]$ every element of $R[s]$ can be uniquely
  expressed as an $R$-linear combination of the elements of $\s{s^0,
    \ldots, s^d}$. Now define $e:R[s]\to R^{k\times k}$ by
  \[
  e \left( \sum r_j s^j \right) = \sum (r_j \otimes I) \cdot
  M^j.
  \]
  This works?
\end{proof}

\begin{proposition}
  If $e:R[s] \to R^{k\times k}$ is a
  $k$-embedding, then $e$ is completely determined by its action on
  $s$. Moreover, if $\ove :R[s]^{n\times n} \to R^{nk\times
    nk}$ extends $e$ then $\ove (s) = I\otimes e(s)$.
\end{proposition}

\begin{proof}
  Every element $t\in R[s]$ can be written as a linear combination
  \[
  t = \sum_{j=0}^{d-1} r_j s^j.
  \]
  Since $e$ is a $k$-embedding we have $e(t) = \sum r_j \cdot
  e(s)^j$. Hence, if two embeddings coincide on $s$ they must actually
  be equal. Now suppose that $\ove $ is an extension $e$. Then $\ove$
  is a $k$-embedding and so $\ove (M) = M\otimes I$ for every $M\in
  R^{n\times n}$. Hence, since $s M = Ms$ in $R[s]^{n\times n}$, we
  have
  \[
  \ove(s) \cdot M\otimes I = M\otimes I \cdot \ove(s)
  \]
  for every $M\in R^{n\times n}$. Thus $\ove(s) = I\otimes e(s)$.
\end{proof}

% --------------------------------------------------------------------
\section{Unitary Restrictions of Embeddings}

\end{document}
